\chapter*{Abstract}
\addcontentsline{toc}{chapter}{Abstract}

\acrfull{puf} is a promising way to redefine how we implement security in modern systems. \acrfull{teropuf} is a recent method that shows good performance with compact design. In this work, we propose two \acrshort{teropuf} implementations on Artix-7 \acrfull{fpga}, the first using one slice per cell and the second using one \acrfull{clb} per cell. The more compact one doesn't reach state of the art performance, while the second gives performance similar to the existing \acrshort{teropuf} implementation on other devices. This shows the limitations of the size of this \acrshort{puf}. We make use of \acrfull{ecc} to further improve the reliability of the response, and we have added a \acrfull{sha} to generate a key that can be used for encryption demonstration.\\

\textbf{Keywords:} \acrlong{puf}; \acrlong{teropuf}; \acrlong{fpga}; Artix-7; Secure hardware design; Cryptographic primitive

\vspace*{\fill}\\
\noindent Oscar Van Slijpe\\
Electronics and Information Technology Engineering\\
Physical layer security: authentication and ciphering using physical unclonable functions on FPGA\\
2022-2023