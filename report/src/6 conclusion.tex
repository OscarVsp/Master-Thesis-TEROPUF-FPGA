\chapter{Conclusion}

%ARCHITECTURE



%SUMMARY OF WORK (problem + concept)

\acrshort{teropuf} is a promising technique that needs further research to cover more devices and possible variations. In particular, to our knowledge, there is no existing implementation of \acrshort{teropuf} on Artix-7 documented in the literature, therefore, we have proposed an implementation.\\

During the design of the \acrshort{puf} (Chapter \ref{ch:impl&chara}), two parameters were identified: the size of the \acrshort{tero} cells and the acquisition time. The discussion on the size of the cell led to two implementations: TERO-4, with \acrshort{tero} cells using 4 \acrshort{lut}s, and TERO-8, with 8 \acrshort{lut}s.\\

The acquisition time is not fixed to the design and has been chosen to have the best performance for the experimental result (Chapter \ref{ch:result}).
The two implementations metrics were studied on a single device, in terms of cell behaviour and performance, for different acquisition times. We found that the TERO-4 implementation produced too many non-oscillating cells and appeared to be unusable (at least on this device), while the TERO-8 worked as expected.\\

Inter-device analysis was performed for TERO-4 and TERO-8 on 33 Basys-3 boards for a $1 \mu s$ acquisition time. The results confirms that the TERO-4 is not usable on most of the devices while the TERO-8 implementation delivers performance in line with other \acrshort{puf} and \acrshort{teropuf} implementations in literature (table~\ref{tab:tero_impls} and table~\ref{tab:artix7_impl}).\\
 
We also demonstrated the use of the \acrshort{sha}-256 and the full demonstration using the Python module.\\

%FUTUR WORK
A meaningful addition to the demonstration would be to have an \acrfull{aes} block in the implementation and to do text (de)-ciphering on the \acrshort{fpga}. This way, the generated key does not have to leave the device, reducing a potential leak.\\
\acrshort{puf} are generally sensitive to factors such as temperature or voltage, and it is important to study their impact on performance to characterise the conditions under which they can be used.
In addition, \cite{tebelmann_side-channel_2019} has studied side-channel possibilities on \acrshort{teropuf} and proposed some countermeasures, that could be added to the design used here.\\
Finally, we choose in this work to limit the \acrshort{tero} cells to one \acrshort{clb} to simplify the routing and maintain a compact implementation. Larger cells could lead to different performances and should be studied.